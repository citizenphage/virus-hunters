% Options for packages loaded elsewhere
\PassOptionsToPackage{unicode}{hyperref}
\PassOptionsToPackage{hyphens}{url}
%
\documentclass[
]{book}
\usepackage{lmodern}
\usepackage{amssymb,amsmath}
\usepackage{ifxetex,ifluatex}
\ifnum 0\ifxetex 1\fi\ifluatex 1\fi=0 % if pdftex
  \usepackage[T1]{fontenc}
  \usepackage[utf8]{inputenc}
  \usepackage{textcomp} % provide euro and other symbols
\else % if luatex or xetex
  \usepackage{unicode-math}
  \defaultfontfeatures{Scale=MatchLowercase}
  \defaultfontfeatures[\rmfamily]{Ligatures=TeX,Scale=1}
\fi
% Use upquote if available, for straight quotes in verbatim environments
\IfFileExists{upquote.sty}{\usepackage{upquote}}{}
\IfFileExists{microtype.sty}{% use microtype if available
  \usepackage[]{microtype}
  \UseMicrotypeSet[protrusion]{basicmath} % disable protrusion for tt fonts
}{}
\makeatletter
\@ifundefined{KOMAClassName}{% if non-KOMA class
  \IfFileExists{parskip.sty}{%
    \usepackage{parskip}
  }{% else
    \setlength{\parindent}{0pt}
    \setlength{\parskip}{6pt plus 2pt minus 1pt}}
}{% if KOMA class
  \KOMAoptions{parskip=half}}
\makeatother
\usepackage{xcolor}
\IfFileExists{xurl.sty}{\usepackage{xurl}}{} % add URL line breaks if available
\IfFileExists{bookmark.sty}{\usepackage{bookmark}}{\usepackage{hyperref}}
\hypersetup{
  pdftitle={2021 Third Year Phage Hunters},
  pdfauthor={Ben Temperton},
  hidelinks,
  pdfcreator={LaTeX via pandoc}}
\urlstyle{same} % disable monospaced font for URLs
\usepackage{longtable,booktabs}
% Correct order of tables after \paragraph or \subparagraph
\usepackage{etoolbox}
\makeatletter
\patchcmd\longtable{\par}{\if@noskipsec\mbox{}\fi\par}{}{}
\makeatother
% Allow footnotes in longtable head/foot
\IfFileExists{footnotehyper.sty}{\usepackage{footnotehyper}}{\usepackage{footnote}}
\makesavenoteenv{longtable}
\usepackage{graphicx}
\makeatletter
\def\maxwidth{\ifdim\Gin@nat@width>\linewidth\linewidth\else\Gin@nat@width\fi}
\def\maxheight{\ifdim\Gin@nat@height>\textheight\textheight\else\Gin@nat@height\fi}
\makeatother
% Scale images if necessary, so that they will not overflow the page
% margins by default, and it is still possible to overwrite the defaults
% using explicit options in \includegraphics[width, height, ...]{}
\setkeys{Gin}{width=\maxwidth,height=\maxheight,keepaspectratio}
% Set default figure placement to htbp
\makeatletter
\def\fps@figure{htbp}
\makeatother
\setlength{\emergencystretch}{3em} % prevent overfull lines
\providecommand{\tightlist}{%
  \setlength{\itemsep}{0pt}\setlength{\parskip}{0pt}}
\setcounter{secnumdepth}{5}
\usepackage{booktabs}
\usepackage[]{natbib}
\bibliographystyle{apalike}

\title{2021 Third Year Phage Hunters}
\author{Ben Temperton}
\date{2021-04-27}

\begin{document}
\maketitle

{
\setcounter{tocdepth}{1}
\tableofcontents
}
\hypertarget{welcome-phage-hunters}{%
\chapter{Welcome Phage Hunters}\label{welcome-phage-hunters}}

\hypertarget{intro}{%
\chapter{Introduction}\label{intro}}

\hypertarget{cpl-plan---week-1}{%
\chapter{CPL Plan - Week 1}\label{cpl-plan---week-1}}

In total, we will have 12 students isolating phages for the CPL. Each student will be assigned a pathogen and will use that to isolate phages from samples crowdsourced by their peers. The first week is spent enriching for phages from their samples. In total there will be 47 samples (plus one negative ctrl).

\hypertarget{monday-10th-may}{%
\section{Monday 10th May}\label{monday-10th-may}}

\begin{itemize}
\tightlist
\item
  Kick off meeting + students given sample packs, with samples to be taken next morning.
\end{itemize}

BT will give a presentation on the work and the steps involved in phage isolation.

\hypertarget{estimated-time-1-hr}{%
\subsection{Estimated time: 1 hr}\label{estimated-time-1-hr}}

\hypertarget{materials-required}{%
\subsection{Materials required}\label{materials-required}}

\begin{itemize}
\tightlist
\item
  36 sample jars
\item
  12 lab pens
\end{itemize}

\hypertarget{tuesday-11th-may}{%
\section{Tuesday 11th May}\label{tuesday-11th-may}}

\begin{itemize}
\tightlist
\item
  Students return to lab with samples
\item
  Samples are transferred to 50 mL falcon tube + centrifuged for 30 mins @ 8000 x g
\item
  Samples are filtered through 0.2 µm syringe filters into fresh 50 mL Falcon tube, then aliquoted into 12 1.5 mL lo-bind microcentrifuge tubes
\item
  Samples are recorded on class shared spreadsheet
\item
  Samples stored O/N at 4 °C.
\end{itemize}

\hypertarget{materials-required-1}{%
\subsection{Materials required}\label{materials-required-1}}

\begin{itemize}
\tightlist
\item
  6 x 50 mL Falcon tube x 12 students (72 total)
\item
  3 x 25 mL syringe with luer lock (36 total)
\item
  3 x 0.2 µm syrige filter (36 total)
\item
  3 x 12 1.5 mL microcentrifuge tubes (432 total)
\item
  microcentrifuge rack
\item
  Centrifuge capable of 8,000 x g for 50 mL Falcon tubes
\end{itemize}

\hypertarget{wednedsay-12th-may}{%
\section{Wednedsay 12th May}\label{wednedsay-12th-may}}

\begin{itemize}
\tightlist
\item
  Students assigned a pathogen
\item
  Students provided a set of samples
\item
  Students aliquot 900 µL of each sample into each well of a 96-well deep well plate
\item
  Students aliquot 500 µL of 3x LB + 30 mM \(MgCl_{2}\) + 30 mM \(CaCl_{2}\) into each well
\item
  Students aliquot 100 µL of O/N pathogen culture into each well.
\item
  Cover and incubate overnight at 30 °C on an orbital shaker.
\item
  Students set up fresh O/N culture of pathogen
\end{itemize}

\hypertarget{materials-required-2}{%
\subsection{Materials required}\label{materials-required-2}}

\begin{itemize}
\tightlist
\item
  10 mL of O/N pathogen culture (200 mL total)
\item
  50 mL of 3x LB + 30 mM \(MgCl_{2}\) + 30 mM \(CaCl_{2}\) (1L total)
\item
  Deep well plate (12 total)
\item
  PCR film (12 total)
\item
  2 sterilins with 10 mL LB + 10 mM \(MgCl_{2}\) + 10 mM \(CaCl_{2}\) (24 total)
\end{itemize}

\hypertarget{thursday-13th-may}{%
\section{Thursday 13th May}\label{thursday-13th-may}}

\begin{itemize}
\tightlist
\item
  Students transfer 200 µL from each well into a 0.2 µm filter plate atop a regular, sterile 96 well plate.
\item
  Plate is spun at 900 x g for 4 mins to transfer filtrate to bottom plate
\item
  Into a fresh 200 µL 96-well plate, students add 190 µL of LB + 10 mM \(MgCl_{2}\) + 10 mM \(CaCl_{2}\) per well
\item
  5 µL of phage lysate from bottom plate is added to each well
\item
  10 µL of O/N host culture added to each well and plate sealed
\item
  Placed in an incubator O/N at 37 °C.
\item
  Students set up fresh O/N culture of pathogen
\end{itemize}

\hypertarget{materials-required-3}{%
\subsection{Materials required}\label{materials-required-3}}

\begin{itemize}
\tightlist
\item
  0.45 µm filter plate (Merck MSHAS4510) (12 total)
\item
  2 x sterile 200 µL 96-well plate (24 total)
\item
  PCR film (12 total)
\item
  2 sterilins with 10 mL LB + 10 mM \(MgCl_{2}\) + 10 mM \(CaCl_{2}\) (24 total)
\end{itemize}

\hypertarget{friday-14th-may}{%
\section{Friday 14th May}\label{friday-14th-may}}

\begin{itemize}
\tightlist
\item
  Students transfer 200 µL from each well into a 0.2 µm filter plate atop a regular, sterile 96 well plate.
\item
  Plate is spun at 900 x g for 4 mins to transfer filtrate to bottom plate
\item
  Bottom plate is covered with PCR film and stored until Monday at 4 °C.
\item
  Students streak pathogen onto agar plate for growth over weekend.
\end{itemize}

\hypertarget{materials-required-4}{%
\subsection{Materials required}\label{materials-required-4}}

\begin{itemize}
\tightlist
\item
  0.45 µm filter plate (12 total)
\item
  1 x sterile 200 µL 96-well plate (12 total)
\item
  PCR film (12 total)
\item
  2 x LB agar plates for bacterial growth
\item
  Inoculating loops
\end{itemize}

\hypertarget{cpl-plan---week-2}{%
\chapter{CPL Plan - Week 2}\label{cpl-plan---week-2}}

In the second week, the students will be performing spot assays and phage purification.

\hypertarget{monday-17th-may}{%
\section{Monday 17th May}\label{monday-17th-may}}

Students will set up four top agar spot plates, with 12 samples per plate (w. neg. ctrl).

\begin{itemize}
\tightlist
\item
  Students will mark plates into quadrants
\item
  Students will add 1 mL of host culture at \$OD\_\{600\} of 0.6 to 3 mL of molten top agar (0.6\% agar in LB + 10 mM \(MgCl_{2}\) + 10 mM \(CaCl_{2}\)), vortex briefly and then pour onto bottom plate
\item
  Once the plates are set, students will spot 5 µL of phage lysate from each sample at 3 per quadrant
\item
  Plates will be incubated O/N at 37 °C for plaque formation
\end{itemize}

\hypertarget{materials-required}{%
\subsection{Materials required}\label{materials-required}}

\begin{itemize}
\tightlist
\item
  10 mL pathogen culture
\item
  5 glass test tubes with lids containing 3 mL molten agar (60 total)
\item
  5 petri dishes with bottom agar (60 total)
\item
  water bath (or heat blocks if the tubes will fit in).
\end{itemize}

\hypertarget{tuesday-18th-may}{%
\section{Tuesday 18th May}\label{tuesday-18th-may}}

Students will pick plaques and perform first round of O/N purification steps. We are going to assume a maximum of 3 phages per student.

\begin{itemize}
\tightlist
\item
  Students will take a photo of their plates to record plaque morphology
\item
  For each plaque, students will core the plaque using a pipette tip and place it in 200 µL of SM buffer in a microcentrifuge tube, before vortexing briefly.
\item
  Students will then set up a dilution series from \(10^{0}\) to \(10^{-11}\) in SM Buffer in a 96 well plate (one row per phage)
\item
  Using the same method as the spot assay, students will spot the 12 dilutions onto a single plate (so one plate per phage).
\item
  Plates will be left O/N for plaques to develop
\end{itemize}

\hypertarget{materials-required-1}{%
\subsection{Materials required}\label{materials-required-1}}

\begin{itemize}
\tightlist
\item
  10 mL pathogen culture
\item
  3 glass test tubes with lids containing 3 mL molten agar (36 total)
\item
  3 petri dishes with bottom agar (36 total)
\item
  water bath (or heat blocks if the tubes will fit in).
\item
  1 sterile 96 well plate (12 total)
\item
  3 lo-bind microcentrifuge tubes (36 total)
\end{itemize}

\hypertarget{wednesday-19th-may}{%
\section{Wednesday 19th May}\label{wednesday-19th-may}}

From each plate, students will pick plaque from the biggest dilution (fewest phages) and perform second round of O/N purification. We are going to assume a maximum of 3 phages per student.

\begin{itemize}
\tightlist
\item
  Students will take a photo of their plates to record plaque morphology
\item
  For each plaque, students will core the plaque using a pipette tip and place it in 200 µL of SM buffer in a microcentrifuge tube, before vortexing briefly.
\item
  Students will then set up a dilution series from \(10^{0}\) to \(10^{-11}\) in SM Buffer in a 96 well plate (one row per phage)
\item
  Using the same method as the spot assay, students will spot the 12 dilutions onto a single plate (so one plate per phage).
\item
  Plates will be left O/N for plaques to develop
\end{itemize}

\hypertarget{materials-required-2}{%
\subsection{Materials required}\label{materials-required-2}}

\begin{itemize}
\tightlist
\item
  10 mL pathogen culture
\item
  3 glass test tubes with lids containing 3 mL molten agar (36 total)
\item
  3 petri dishes with bottom agar (36 total)
\item
  water bath (or heat blocks if the tubes will fit in).
\item
  1 sterile 96 well plate (12 total)
\item
  3 lo-bind microcentrifuge tubes (36 total)
\end{itemize}

\hypertarget{thursday-20th-may}{%
\section{Thursday 20th May}\label{thursday-20th-may}}

From each plate, students will pick plaque from the biggest dilution (fewest phages) and perform third round of O/N purification. We are going to assume a maximum of 3 phages per student.

\begin{itemize}
\tightlist
\item
  Students will take a photo of their plates to record plaque morphology
\item
  For each plaque, students will core the plaque using a pipette tip and place it in 200 µL of SM buffer in a microcentrifuge tube, before vortexing briefly.
\item
  Students will then set up a dilution series from \(10^{0}\) to \(10^{-11}\) in SM Buffer in a 96 well plate (one row per phage)
\item
  Using the same method as the spot assay, students will spot the 12 dilutions onto a single plate (so one plate per phage).
\item
  Plates will be left O/N for plaques to develop
\item
  Students will set up O/N cultures of host
\end{itemize}

\hypertarget{materials-required-3}{%
\subsection{Materials required}\label{materials-required-3}}

\begin{itemize}
\tightlist
\item
  10 mL pathogen culture
\item
  3 glass test tubes with lids containing 3 mL molten agar (36 total)
\item
  3 petri dishes with bottom agar (36 total)
\item
  water bath (or heat blocks if the tubes will fit in).
\item
  1 sterile 96 well plate (12 total)
\item
  3 lo-bind microcentrifuge tubes (36 total)
\item
  2 sterilins containing 10 mL LB + 10 mM \(MgCl_{2}\) + 10 mM \(CaCl_{2}\) (24 total)
\end{itemize}

\hypertarget{friday-21st-may}{%
\section{Friday 21st May}\label{friday-21st-may}}

From each plate, students will pick plaque from the biggest dilution (fewest phages) and bulk it up on hosts overnight.

\begin{itemize}
\tightlist
\item
  Students will take a photo of their plates to record plaque morphology. These will also be used to determine approximate PFU counts.
\item
  Students will pick plaque from the biggest dilution (fewest phages) and add it to 50 mL of LB + 10 mM \(MgCl_{2}\) + 10 mM \(CaCl_{2}\), amended with 1 mL of O/N host culture.
\item
  Cultures will be grown O/N at 30 °C
\end{itemize}

The next day (Saturday), BT and team will go in and centrifuge the samples down and prepare for each phage 1 x 50 mL tubes of lysate, filtered through a 0.2 µm syringe filter. We will prepare 3 x 2 mL phage stocks in acid washed, autoclaved amber glass vials.

\textbf{200 µL of each phage filtrate will be provided to the imaging centre.}

\hypertarget{materials-required-4}{%
\subsection{Materials required}\label{materials-required-4}}

\begin{itemize}
\tightlist
\item
  3 x 50 mL LB + 10 mM \(MgCl_{2}\) + 10 mM \(CaCl_{2}\) (2L total)
\item
  3 x 50 mL falcon tubes (36 total)
\item
  3 x 0.2 µm syringe filter (36 total)
\item
  3 x 25 mL syringe with luer lock (36 total).
\item
  3 x 2 mL amber glass vials
\end{itemize}

\hypertarget{cpl-plan---week-3}{%
\chapter{CPL Plan - Week 3}\label{cpl-plan---week-3}}

In the third week, the students will be performing DNA extractions on phage DNA

\hypertarget{monday-24th-may}{%
\section{Monday 24th May}\label{monday-24th-may}}

Students will test their lysate for estimating PFU and begin phage DNA precipitation

\begin{itemize}
\tightlist
\item
  Students will prepare a dilution series of their phages as before and spot them in triplicate across 3 plates
\item
  Students will transfer 30 mL of lysate to a fresh Falcon tube
\item
  Students will add 15 µL of nuclease solution and incubate at 37 °C for 30 mins
\item
  Students will then add 15 mL of precipitant solution to each tube and mix gently by inversion
\item
  Samples will be incubated at 4 °C O/N
\end{itemize}

\hypertarget{materials-required}{%
\subsection{Materials required}\label{materials-required}}

\begin{itemize}
\tightlist
\item
  10 mL pathogen culture
\item
  3 glass test tubes with lids containing 3 mL molten agar (36 total)
\item
  3 petri dishes with bottom agar (36 total)
\item
  water bath (or heat blocks if the tubes will fit in).
\item
  1 sterile 96 well plate (12 total)
\item
  3 x Falcon tube (36 total)
\item
  50 µL of nuclease solution (600 µL total)
\item
  50 mL of precipitant solution (600 mL total)
\end{itemize}

\hypertarget{tuesday-25th-may}{%
\section{Tuesday 25th May}\label{tuesday-25th-may}}

Students will extract DNA using the Promega Wizard kit using \href{https://cpt.tamu.edu/wordpress/wp-content/uploads/2011/12/Phage-DNA-extraction-modified-Wizard-method-07-12-2011.pdf}{this protocol}

Depending on how many phages we isolate, we are anticipating sending two per student for sequencing, with the remaining extractions kept back for future analyses.
\#\#\# Materials required
* 3 x 500 µL resuspension buffer (5 mM \(MgSO_{4}\)) (18 mL total)
* 2 x Promega Wizard Kits.

\hypertarget{precipitant-solution}{%
\section{Precipitant solution}\label{precipitant-solution}}

This is a ready-mixed 30\% w/v PEG-8000, 3 M NaCl solution for adding to phage lysate in a 1:2 ratio precipitant:lysate (10\%PEG-8000, 1 M NaCl final conc.)

\begin{enumerate}
\def\labelenumi{\arabic{enumi}.}
\tightlist
\item
  In a sterilized 500 mL bottle add 330 mL of autoclaved MilliQ and 105 g of NaCl and dissolve.
\item
  Add 180g of PEG8000, cap bottle and shake.
\item
  Incubate bottle in a 60 °C waterbath for 3 hours, shaking occasionally
\item
  Remove and let cool to RT, shaking occasionally
\item
  Add autoclaved MilliQ to 600 mL and store at RT.
\end{enumerate}

  \bibliography{book.bib,packages.bib}

\end{document}
